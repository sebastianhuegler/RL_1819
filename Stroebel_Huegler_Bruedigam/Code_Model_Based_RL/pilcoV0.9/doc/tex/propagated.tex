
% This LaTeX was auto-generated from an M-file by MATLAB.
% To make changes, update the M-file and republish this document.



    
    
      \subsection{propagated.m}

\begin{par}
\textbf{Summary:} Propagate the state distribution one time step forward            with derivatives
\end{par} \vspace{1em}

\begin{verbatim}function [Mnext, Snext, dMdm, dSdm, dMds, dSds, dMdp, dSdp] = ...
  propagated(m, s, plant, dynmodel, policy)\end{verbatim}
    \begin{par}
\textbf{Input arguments:}
\end{par} \vspace{1em}
\begin{verbatim}m                 mean of the state distribution at time t           [D x 1]
s                 covariance of the state distribution at time t     [D x D]
plant             plant structure
dynmodel          dynamics model structure
policy            policy structure\end{verbatim}
\begin{par}
\textbf{Output arguments:}
\end{par} \vspace{1em}
\begin{verbatim}Mnext             predicted mean at time t+1                         [E x 1]
Snext             predicted covariance at time t+1                   [E x E]
dMdm              output mean wrt input mean                         [E x D]
dMds              output mean wrt input covariance matrix         [E  x D*D]
dSdm              output covariance matrix wrt input mean        [E*E x  D ]
dSds              output cov wrt input cov                       [E*E x D*D]
dMdp              output mean wrt policy parameters                  [E x P]
dSdp              output covariance matrix wrt policy parameters  [E*E x  P]\end{verbatim}
\begin{verbatim}where P is the number of policy parameters.\end{verbatim}
\begin{par}
Copyright (C) 2008-2013 by Marc Deisenroth, Andrew McHutchon, Joe Hall, Henrik Ohlsson, and Carl Edward Rasmussen.
\end{par} \vspace{1em}
\begin{par}
Last modified: 2013-01-23
\end{par} \vspace{1em}


\subsection*{High-Level Steps} 

\begin{enumerate}
\setlength{\itemsep}{-1ex}
   \item Augment state distribution with trigonometric functions
   \item Compute distribution of the control signal
   \item Compute dynamics-GP prediction
   \item Compute distribution of the next state
\end{enumerate}

\begin{lstlisting}
function [Mnext, Snext, dMdm, dSdm, dMds, dSds, dMdp, dSdp] = ...
  propagated(m, s, plant, dynmodel, policy)
\end{lstlisting}


\subsection*{Code} 


\begin{lstlisting}
if nargout <= 2                                  % just predict, no derivatives
  [Mnext, Snext] = propagate(m, s, plant, dynmodel, policy);
  return
end

angi = plant.angi; poli = plant.poli; dyni = plant.dyni; difi = plant.difi;

D0 = length(m);                                        % size of the input mean
D1 = D0 + 2*length(angi);          % length after mapping all angles to sin/cos
D2 = D1 + length(policy.maxU);          % length after computing control signal
D3 = D2 + D0;                                         % length after predicting
M = zeros(D3,1); M(1:D0) = m; S = zeros(D3); S(1:D0,1:D0) = s;   % init M and S

Mdm = [eye(D0); zeros(D3-D0,D0)]; Sdm = zeros(D3*D3,D0);
Mds = zeros(D3,D0*D0); Sds = kron(Mdm,Mdm);
X = reshape(1:D3*D3,[D3 D3]); XT = X'; Sds = (Sds + Sds(XT(:),:))/2;
X = reshape(1:D0*D0,[D0 D0]); XT = X'; Sds = (Sds + Sds(:,XT(:)))/2;

% 1) Augment state distribution with trigonometric functions ------------------
i = 1:D0; j = 1:D0; k = D0+1:D1;
[M(k) S(k,k) C mdm sdm Cdm mds sds Cds] = gTrig(M(i), S(i,i), angi);

[S Mdm Mds Sdm Sds] = ...
  fillIn(S,C,mdm,sdm,Cdm,mds,sds,Cds,Mdm,Sdm,Mds,Sds,[ ],[ ],[ ],i,j,k,D3);

mm=zeros(D1,1); mm(i)=M(i); ss(i,i)=S(i,i)+diag(exp(2*dynmodel.hyp(end,:))/2);
[mm(k), ss(k,k) C] = gTrig(mm(i), ss(i,i), angi);     % noisy state measurement
q = ss(j,i)*C; ss(j,k) = q; ss(k,j) = q';

% 2) Compute distribution of the control signal -------------------------------
i = poli; j = 1:D1; k = D1+1:D2;
[M(k) S(k,k) C mdm sdm Cdm mds sds Cds Mdp Sdp Cdp] = ...
  policy.fcn(policy, mm(i), ss(i,i));

[S Mdm Mds Sdm Sds Mdp Sdp] = ...
  fillIn(S,C,mdm,sdm,Cdm,mds,sds,Cds,Mdm,Sdm,Mds,Sds,Mdp,Sdp,Cdp,i,j,k,D3);

% 3) Compute distribution of the change in state ------------------------------
ii = [dyni D1+1:D2]; j = 1:D2;
if isfield(dynmodel,'sub'), Nf = length(dynmodel.sub); else Nf = 1; end
for n=1:Nf                               % potentially multiple dynamics models
  [dyn i k] = sliceModel(dynmodel,n,ii,D1,D2,D3); j = setdiff(j,k);

  [M(k) S(k,k) C mdm sdm Cdm mds sds Cds] = dyn.fcn(dyn, M(i), S(i,i));

  [S Mdm Mds Sdm Sds Mdp Sdp] = ...
    fillIn(S,C,mdm,sdm,Cdm,mds,sds,Cds,Mdm,Sdm,Mds,Sds,Mdp,Sdp,[ ],i,j,k,D3);

  j = [j k];                                   % update 'previous' state vector
end

% 4) Compute distribution of the next state -----------------------------------
P = [zeros(D0,D2) eye(D0)]; P(difi,difi) = eye(length(difi));  P = sparse( P);
Mnext = P*M; Snext = P*S*P'; Snext = (Snext+Snext')/2;

PP = kron(P,P);
dMdm =  P*Mdm; dMds =  P*Mds; dMdp =  P*Mdp;
dSdm = PP*Sdm; dSds = PP*Sds; dSdp = PP*Sdp;

X = reshape(1:D0*D0,[D0 D0]); XT = X';                          % symmetrize dS
dSdm = (dSdm + dSdm(XT(:),:))/2; dMds = (dMds + dMds(:,XT(:)))/2;
dSds = (dSds + dSds(XT(:),:))/2; dSds = (dSds + dSds(:,XT(:)))/2;
dSdp = (dSdp + dSdp(XT(:),:))/2;


% A1) Separate multiple dynamics models ---------------------------------------
\end{lstlisting}

\begin{lstlisting}
function [dyn i k] = sliceModel(dynmodel,n,ii,D1,D2,D3) % separate sub-dynamics
if isfield(dynmodel,'sub')
  dyn = dynmodel.sub{n}; do = dyn.dyno; D = length(ii)+D1-D2;
  if isfield(dyn,'dyni'), di=dyn.dyni; else di=[]; end
  if isfield(dyn,'dynu'), du=dyn.dynu; else du=[]; end
  if isfield(dyn,'dynj'), dj=dyn.dynj; else dj=[]; end
  i = [ii(di) D1+du D2+dj]; k = D2+do;
  dyn.inputs = [dynmodel.inputs(:,[di D+du]) dynmodel.target(:,dj)];   % inputs
  dyn.target = dynmodel.target(:,do);                                 % targets
else
  dyn = dynmodel; k = D2+1:D3; i = ii;
end

% A2) Apply chain rule and fill out cross covariance terms --------------------
function [S Mdm Mds Sdm Sds Mdp Sdp] = ...
  fillIn(S,C,mdm,sdm,Cdm,mds,sds,Cds,Mdm,Sdm,Mds,Sds,Mdp,Sdp,dCdp,i,j,k,D)

if isempty(k), return; end

X = reshape(1:D*D,[D D]); XT = X';                         % vectorized indices
I=0*X; I(i,i)=1; ii=X(I==1)'; I=0*X; I(k,k)=1; kk=X(I==1)';
I=0*X; I(j,i)=1; ji=X(I==1)'; I=0*X; I(j,k)=1; jk=X(I==1)'; kj=XT(I==1)';

Mdm(k,:)  = mdm*Mdm(i,:) + mds*Sdm(ii,:);                           % chainrule
Mds(k,:)  = mdm*Mds(i,:) + mds*Sds(ii,:);
Sdm(kk,:) = sdm*Mdm(i,:) + sds*Sdm(ii,:);
Sds(kk,:) = sdm*Mds(i,:) + sds*Sds(ii,:);
dCdm      = Cdm*Mdm(i,:) + Cds*Sdm(ii,:);
dCds      = Cdm*Mds(i,:) + Cds*Sds(ii,:);
if isempty(dCdp) && nargout > 5
  Mdp(k,:)  = mdm*Mdp(i,:) + mds*Sdp(ii,:);
  Sdp(kk,:) = sdm*Mdp(i,:) + sds*Sdp(ii,:);
  dCdp      = Cdm*Mdp(i,:) + Cds*Sdp(ii,:);
elseif nargout > 5
  aa = length(k); bb = aa^2; cc = numel(C);
  mdp = zeros(D,size(Mdp,2)); sdp = zeros(D*D,size(Mdp,2));
  mdp(k,:)  = reshape(Mdp,aa,[]); Mdp = mdp;
  sdp(kk,:) = reshape(Sdp,bb,[]); Sdp = sdp;
  Cdp       = reshape(dCdp,cc,[]); dCdp = Cdp;
end

q = S(j,i)*C; S(j,k) = q; S(k,j) = q';                           % off-diagonal
SS = kron(eye(length(k)),S(j,i)); CC = kron(C',eye(length(j)));
Sdm(jk,:) = SS*dCdm + CC*Sdm(ji,:); Sdm(kj,:) = Sdm(jk,:);
Sds(jk,:) = SS*dCds + CC*Sds(ji,:); Sds(kj,:) = Sds(jk,:);
if nargout > 5
  Sdp(jk,:) = SS*dCdp + CC*Sdp(ji,:); Sdp(kj,:) = Sdp(jk,:);
end
\end{lstlisting}
