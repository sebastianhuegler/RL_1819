
% This LaTeX was auto-generated from an M-file by MATLAB.
% To make changes, update the M-file and republish this document.



    
    

\subsection*{error\_ellipse.m} 

\begin{par}
ERROR\_ELLIPSE - plot an error ellipse, or ellipsoid, defining confidence region    ERROR\_ELLIPSE(C22) - Given a 2x2 covariance matrix, plot the    associated error ellipse, at the origin. It returns a graphics handle    of the ellipse that was drawn.
\end{par} \vspace{1em}

\begin{verbatim}  ERROR_ELLIPSE(C33) - Given a 3x3 covariance matrix, plot the
  associated error ellipsoid, at the origin, as well as its projections
  onto the three axes. Returns a vector of 4 graphics handles, for the
  three ellipses (in the X-Y, Y-Z, and Z-X planes, respectively) and for
  the ellipsoid.\end{verbatim}
    
\begin{verbatim}  ERROR_ELLIPSE(C,MU) - Plot the ellipse, or ellipsoid, centered at MU,
  a vector whose length should match that of C (which is 2x2 or 3x3).\end{verbatim}
    
\begin{verbatim}  ERROR_ELLIPSE(...,'Property1',Value1,'Name2',Value2,...) sets the
  values of specified properties, including:
    'C' - Alternate method of specifying the covariance matrix
    'mu' - Alternate method of specifying the ellipse (-oid) center
    'conf' - A value betwen 0 and 1 specifying the confidence interval.
      the default is 0.5 which is the 50% error ellipse.
    'scale' - Allow the plot the be scaled to difference units.
    'style' - A plotting style used to format ellipses.
    'clip' - specifies a clipping radius. Portions of the ellipse, -oid,
      outside the radius will not be shown.\end{verbatim}
    
\begin{verbatim}  NOTES: C must be positive definite for this function to work properly.\end{verbatim}
    \begin{par}
by AJ Johnson \begin{verbatim}http://www.mathworks.de/matlabcentral/fileexchange/4705\end{verbatim}
\end{par} \vspace{1em}

\begin{lstlisting}
function h=error_ellipse(varargin)
\end{lstlisting}


\subsection*{Code} 


\begin{lstlisting}
default_properties = struct(...
  'C', [], ... % The covaraince matrix (required)
  'mu', [], ... % Center of ellipse (optional)
  'conf', 0.95, ... % Percent confidence/100
  'scale', 1, ... % Scale factor, e.g. 1e-3 to plot m as km
  'style', '', ...  % Plot style
  'clip', inf); % Clipping radius

if length(varargin) >= 1 & isnumeric(varargin{1})
  default_properties.C = varargin{1};
  varargin(1) = [];
end

if length(varargin) >= 1 & isnumeric(varargin{1})
  default_properties.mu = varargin{1};
  varargin(1) = [];
end

if length(varargin) >= 1 & isnumeric(varargin{1})
  default_properties.conf = varargin{1};
  varargin(1) = [];
end

if length(varargin) >= 1 & isnumeric(varargin{1})
  default_properties.scale = varargin{1};
  varargin(1) = [];
end

if length(varargin) >= 1 & ~ischar(varargin{1})
  error('Invalid parameter/value pair arguments.')
end

prop = getopt(default_properties, varargin{:});
C = prop.C;

if isempty(prop.mu)
  mu = zeros(length(C),1);
else
  mu = prop.mu;
end

conf = prop.conf;
scale = prop.scale;
style = prop.style;

if conf <= 0 | conf >= 1
  error('conf parameter must be in range 0 to 1, exclusive')
end

[r,c] = size(C);
if r ~= c | (r ~= 2 & r ~= 3)
  error(['Don''t know what to do with ',num2str(r),'x',num2str(c),' matrix'])
end

x0=mu(1);
y0=mu(2);

% Compute quantile for the desired percentile
k = sqrt(qchisq(conf,r)); % r is the number of dimensions (degrees of freedom)

hold_state = get(gca,'nextplot');

if r==3 & c==3
  z0=mu(3);

  % Make the matrix has positive eigenvalues - else it's not a valid covariance matrix!
  if any(eig(C) <=0)
    error('The covariance matrix must be positive definite (it has non-positive eigenvalues)')
  end

  % C is 3x3; extract the 2x2 matricies, and plot the associated error
  % ellipses. They are drawn in space, around the ellipsoid; it may be
  % preferable to draw them on the axes.
  Cxy = C(1:2,1:2);
  Cyz = C(2:3,2:3);
  Czx = C([3 1],[3 1]);

  [x,y,z] = getpoints(Cxy,prop.clip);
  h1=plot3(x0+k*x,y0+k*y,z0+k*z,prop.style,'linewidth',2);hold on
  [y,z,x] = getpoints(Cyz,prop.clip);
  h2=plot3(x0+k*x,y0+k*y,z0+k*z,prop.style,'linewidth',2);hold on
  [z,x,y] = getpoints(Czx,prop.clip);
  h3=plot3(x0+k*x,y0+k*y,z0+k*z,prop.style,'linewidth',2);hold on


  [eigvec,eigval] = eig(C);

  [X,Y,Z] = ellipsoid(0,0,0,1,1,1);
  XYZ = [X(:),Y(:),Z(:)]*sqrt(eigval)*eigvec';

  X(:) = scale*(k*XYZ(:,1)+x0);
  Y(:) = scale*(k*XYZ(:,2)+y0);
  Z(:) = scale*(k*XYZ(:,3)+z0);
  h4=surf(X,Y,Z);
  colormap gray
  alpha(0.3)
  camlight
  if nargout
    h=[h1 h2 h3 h4];
  end
elseif r==2 & c==2
  % Make the matrix has positive eigenvalues - else it's not a valid covariance matrix!
  if any(eig(C) <=0)
    error('The covariance matrix must be positive definite (it has non-positive eigenvalues)')
  end

  [x,y,z] = getpoints(C,prop.clip);
  h1=plot(scale*(x0+k*x),scale*(y0+k*y),prop.style,'linewidth',2);
  set(h1,'zdata',z+1)
  if nargout
    h=h1;
  end
else
  error('C (covaraince matrix) must be specified as a 2x2 or 3x3 matrix)')
end
%axis equal

set(gca,'nextplot',hold_state);

%---------------------------------------------------------------
% getpoints - Generate x and y points that define an ellipse, given a 2x2
%   covariance matrix, C. z, if requested, is all zeros with same shape as
%   x and y.
\end{lstlisting}

\begin{lstlisting}
function [x,y,z] = getpoints(C,clipping_radius)

n=100; % Number of points around ellipse
p=0:pi/n:2*pi; % angles around a circle

[eigvec,eigval] = eig(C); % Compute eigen-stuff
xy = [cos(p'),sin(p')] * sqrt(eigval) * eigvec'; % Transformation
x = xy(:,1);
y = xy(:,2);
z = zeros(size(x));

% Clip data to a bounding radius
if nargin >= 2
  r = sqrt(sum(xy.^2,2)); % Euclidian distance (distance from center)
  x(r > clipping_radius) = nan;
  y(r > clipping_radius) = nan;
  z(r > clipping_radius) = nan;
end

%---------------------------------------------------------------
function x=qchisq(P,n)
% QCHISQ(P,N) - quantile of the chi-square distribution.
if nargin<2
  n=1;
end

s0 = P==0;
s1 = P==1;
s = P>0 & P<1;
x = 0.5*ones(size(P));
x(s0) = -inf;
x(s1) = inf;
x(~(s0|s1|s))=nan;

for ii=1:14
  dx = -(pchisq(x(s),n)-P(s))./dchisq(x(s),n);
  x(s) = x(s)+dx;
  if all(abs(dx) < 1e-6)
    break;
  end
end

%---------------------------------------------------------------
function F=pchisq(x,n)
% PCHISQ(X,N) - Probability function of the chi-square distribution.
if nargin<2
  n=1;
end
F=zeros(size(x));

if rem(n,2) == 0
  s = x>0;
  k = 0;
  for jj = 0:n/2-1;
    k = k + (x(s)/2).^jj/factorial(jj);
  end
  F(s) = 1-exp(-x(s)/2).*k;
else
  for ii=1:numel(x)
    if x(ii) > 0
      F(ii) = quadl(@dchisq,0,x(ii),1e-6,0,n);
    else
      F(ii) = 0;
    end
  end
end

%---------------------------------------------------------------
function f=dchisq(x,n)
% DCHISQ(X,N) - Density function of the chi-square distribution.
if nargin<2
  n=1;
end
f=zeros(size(x));
s = x>=0;
f(s) = x(s).^(n/2-1).*exp(-x(s)/2)./(2^(n/2)*gamma(n/2));

%---------------------------------------------------------------
function properties = getopt(properties,varargin)
%GETOPT - Process paired optional arguments as 'prop1',val1,'prop2',val2,...
%
%   getopt(properties,varargin) returns a modified properties structure,
%   given an initial properties structure, and a list of paired arguments.
%   Each argumnet pair should be of the form property_name,val where
%   property_name is the name of one of the field in properties, and val is
%   the value to be assigned to that structure field.
%
%   No validation of the values is performed.
%
% EXAMPLE:
%   properties = struct('zoom',1.0,'aspect',1.0,'gamma',1.0,'file',[],'bg',[]);
%   properties = getopt(properties,'aspect',0.76,'file','mydata.dat')
% would return:
%   properties =
%         zoom: 1
%       aspect: 0.7600
%        gamma: 1
%         file: 'mydata.dat'
%           bg: []
%
% Typical usage in a function:
%   properties = getopt(properties,varargin{:})


% Process the properties (optional input arguments)
prop_names = fieldnames(properties);
TargetField = [];
for ii=1:length(varargin)
  arg = varargin{ii};
  if isempty(TargetField)
    if ~ischar(arg)
      error('Propery names must be character strings');
    end
    f = find(strcmp(prop_names, arg));
    if length(f) == 0
      error('%s ',['invalid property ''',arg,'''; must be one of:'],prop_names{:});
    end
    TargetField = arg;
  else
    % properties.(TargetField) = arg; % Ver 6.5 and later only
    properties = setfield(properties, TargetField, arg); % Ver 6.1 friendly
    TargetField = '';
  end
end
if ~isempty(TargetField)
  error('Property names and values must be specified in pairs.');
end
\end{lstlisting}
