
% This LaTeX was auto-generated from an M-file by MATLAB.
% To make changes, update the M-file and republish this document.



    
    

\section*{minimize.m} 

\begin{par}
minimize.m - minimize a smooth differentiable multivariate function using LBFGS (Limited memory LBFGS) or CG (Conjugate Gradients) Usage: [X, fX, i] = minimize(X, F, p, other, ... ) where   X    is an initial guess (any type: vector, matrix, cell array, struct)   F    is the objective function (function pointer or name)   p    parameters - if p is a number, it corresponds to p.length below     p.length     allowed 1) \# linesearches or 2) if -ve minus \# func evals     p.method     minimization method, 'BFGS', 'LBFGS' or 'CG'     p.verbosity  0 quiet, 1 line, 2 line + warnings (default), 3 graphical     p.mem        number of directions used in LBFGS (default 100)   other, ...     other parameters, passed to the function F   X     returned minimizer   fX    vector of function values showing minimization progress   i     final number of linesearches or function evaluations The function F must take the following syntax [f, df] = F(X, other, ...) where f is the function value and df its partial derivatives. The types of X and df must be identical (vector, matrix, cell array, struct, etc).
\end{par} \vspace{1em}
\begin{par}
Copyright (C) 1996 - 2011 by Carl Edward Rasmussen, 2013-02-12.
\end{par} \vspace{1em}

\begin{lstlisting}
% Permission is hereby granted, free of charge, to any person OBTAINING A COPY
% OF THIS SOFTWARE AND ASSOCIATED DOCUMENTATION FILES (THE "SOFTWARE"), TO DEAL
% IN THE SOFTWARE WITHOUT RESTRICTION, INCLUDING WITHOUT LIMITATION THE RIGHTS
% to use, copy, modify, merge, publish, distribute, sublicense, and/or sell
% copies of the Software, and to permit persons to whom the Software is
% furnished to do so, subject to the following conditions:
%
% The above copyright notice and this permission notice shall be included in
% all copies or substantial portions of the Software.
%
% THE SOFTWARE IS PROVIDED "AS IS", WITHOUT WARRANTY OF ANY KIND, EXPRESS OR
% IMPLIED, INCLUDING BUT NOT LIMITED TO THE WARRANTIES OF MERCHANTABILITY,
% FITNESS FOR A PARTICULAR PURPOSE AND NONINFRINGEMENT. IN NO EVENT SHALL THE
% AUTHORS OR COPYRIGHT HOLDERS BE LIABLE FOR ANY CLAIM, DAMAGES OR OTHER
% LIABILITY, WHETHER IN AN ACTION OF CONTRACT, TORT OR OTHERWISE, ARISING FROM,
% OUT OF OR IN CONNECTION WITH THE SOFTWARE OR THE USE OR OTHER DEALINGS IN THE
% SOFTWARE.

function [X, fX, i, p] = minimize(X, F, p, varargin)
if isnumeric(p), p = struct('length', p); end             % convert p to struct
if p.length > 0, p.S = 'linesearch #'; else p.S = 'function evaluation #'; end;
x = unwrap(X);                                % convert initial guess to vector
if ~isfield(p,'method'), if length(x) > 1000, p.method = @LBFGS;
                         else p.method = @BFGS; end; end   % set default method
if ~isfield(p,'verbosity'), p.verbosity = 1; end   % default 1 line text output
if ~isfield(p,'MFEPLS'), p.MFEPLS = 10; end    % Max Func Evals Per Line Search
if ~isfield(p,'MSR'), p.MSR = 100; end                % Max Slope Ratio default
if isfield(p,'fh')&&~isempty(p.fh)&&ishandle(p.fh); clf(p.fh); end
f(F, X, varargin{:});                                   % set up the function f
[fx, dfx] = f(x);                      % initial function value and derivatives
if ~isscalar(fx); error('Objective function value must be a scalar'); end
if any(isnan(fx+dfx)) || any(isinf(fx+dfx));
  error('Evaluating objective function with initial parameters returns NaN or Inf');
end
if p.verbosity, printf('Initial Function Value %4.6e\n', fx); end
if p.verbosity > 2,
  ah = ahandles(p);
  hold(ah(1),'on'); xlabel(ah(1),p.S); ylabel(ah(1),'function value');
  plot(ah(1),p.length < 0, fx, '+'); drawnow;
end
[x, fX, i, p] = feval(p.method, x, fx, dfx, p);  % minimize using direction method
X = rewrap(X, x);                   % convert answer to original representation
if p.verbosity, printf('\n'); end

function [x, fx, i, p] = CG(x0, fx0, dfx0, p)
if ~isfield(p, 'SIG'), p.SIG = 0.1; end       % default for line search quality
i = p.length < 0; ok = 1;                         % initialize resource counter
r = -dfx0; s = -r'*r; b = -1/(s-1); bs = -1; fx = fx0;       % steepest descent
while i < abs(p.length)
  b = b*bs/min(b*s,bs/p.MSR);    % suitable initial step size using slope ratio
  b = min(b,1/norm(r));                    % limit step size in parameter space
  b = max(b,1e-7/norm(r));
  [x, b, fx0, dfx, i] = lineSearch(x0, fx0, dfx0, r, s, b, i, p);
  if i < 0                                              % if line search failed
    i = -i; if ok, ok = 0; r = -dfx; else break; end     % try steepest or stop
  else
    ok = 1; bs = b*s;        % record step times slope (for slope ratio method)
    r = (dfx'*(dfx-dfx0))/(dfx0'*dfx0)*r - dfx;             % Polack-Ribiere CG
  end
  s = r'*dfx; if s >= 0, r = -dfx; s = r'*dfx; ok = 0; end  % slope must be -ve
  x0 = x; dfx0 = dfx; fx = [fx; fx0];        % replace old values with new ones
end

function [x, fx, i, p] = BFGS(x0, fx0, dfx0, p)
if ~isfield(p, 'SIG'), p.SIG = 0.5; end       % default for line search quality
i = p.length < 0; ok = 1;                         % initialize resource counter
x = x0; fx = fx0; r = -dfx0; s = -r'*r; b = -1/(s-1); H = eye(length(x0));
while i < abs(p.length)
 b = min(b,1/norm(r));                    % limit step size in parameter space
 b = max(b,1e-7/norm(r));
  [x, b, fx0, dfx, i] = lineSearch(x0, fx0, dfx0, r, s, b, i, p);
  if i < 0
    i = -i; if ok, ok = 0; else break; end;              % try steepest or stop
  else
    ok = 1; t = x - x0; y = dfx - dfx0; ty = t'*y; Hy = H*y;
    H = H + (ty+y'*Hy)/ty^2*t*t' - 1/ty*Hy*t' - 1/ty*t*Hy';       % BFGS update
  end
  r = -H*dfx; s = r'*dfx; x0 = x; dfx0 = dfx; fx = [fx; fx0];
  p.H = H;

end

function [x, fx, i, p] = LBFGS(x0, fx0, dfx0, p);
if ~isfield(p, 'SIG'), p.SIG = 0.5; end       % default for line search quality
n = length(x0); k = 0; ok = 1; x = x0; fx = fx0; bs = -1/p.MSR;
if isfield(p, 'mem'), m = p.mem; else m = min(100, n); end    % set memory size
a = zeros(1, m); t = zeros(n, m); y = zeros(n, m);            % allocate memory
i = p.length < 0;                                 % initialize resource counter
while i < abs(p.length)
  q = dfx0;
  for j = rem(k-1:-1:max(0,k-m),m)+1
    a(j) = t(:,j)'*q/rho(j); q = q-a(j)*y(:,j);
  end
  if k == 0, r = -q/(q'*q); else r = -t(:,j)'*y(:,j)/(y(:,j)'*y(:,j))*q; end
  for j = rem(max(0,k-m):k-1,m)+1
    r = r-t(:,j)*(a(j)+y(:,j)'*r/rho(j));
  end
  s = r'*dfx0; if s >= 0, r = -dfx0; s = r'*dfx0; k = 0; ok = 0; end
  b = bs/min(bs,s/p.MSR);              % suitable initial step size (usually 1)
  b = min(b,1/norm(r));                    % limit step size in parameter space
  b = max(b,1e-7/norm(r));
  if isnan(r) | isinf(r)                                % if nonsense direction
    i = -i;                                              % try steepest or stop
  else
    [x, b, fx0, dfx, i] = lineSearch(x0, fx0, dfx0, r, s, b, i, p);
  end
  if i < 0                                              % if line search failed
    i = -i; if ok, ok = 0; k = 0; else break; end        % try steepest or stop
  else
    j = rem(k,m)+1; t(:,j) = x-x0; y(:,j) = dfx-dfx0; rho(j) = t(:,j)'*y(:,j);
    ok = 1; k = k+1; bs = b*s;
  end
  x0 = x; dfx0 = dfx; fx = [fx; fx0];                  % replace and add values
end

function [x, a, fx, df, i] = lineSearch(x0, f0, df0, d, s, a, i, p)
if p.length < 0, LIMIT = min(p.MFEPLS, -i-p.length); else LIMIT = p.MFEPLS; end
p0.x = 0.0; p0.f = f0; p0.df = df0; p0.s = s; p1 = p0;         % init p0 and p1
j = 0; p3.x = a; wp(p0, p.SIG, 0);         % set step & Wolfe-Powell conditions
if p.verbosity > 2
  A = [-a a]/5; nd = norm(d); ah = ahandles(p);
  hold(ah(2),'off'); plot(ah(2),0, f0, 'k+'); hold(ah(2),'on'); plot(ah(2),nd*A, f0+s*A, 'k-');
  xlabel(ah(2),'distance in line search direction'); ylabel(ah(2),'function value');
end
while 1                               % keep extrapolating as long as necessary
  ok = 0;
  while ~ok && j < LIMIT
    try           % try, catch and bisect to safeguard extrapolation evaluation
      j = j+1; [p3.f p3.df] = f(x0+p3.x*d); p3.s = p3.df'*d; ok = 1;
      if isnan(p3.f+p3.s) || isinf(p3.f+p3.s)
        error('Objective function returned Inf or NaN','');
      end;
    catch
      if p.verbosity > 1, printf('\n'); warning(lasterr); end % warn or silence
      p3.x = (p1.x+p3.x)/2; ok = 0; p3.f = NaN;  p3.s = NaN;% bisect, and retry
    end
  end
  if p.verbosity > 2
    ah = ahandles(p); hold(ah(2),'on');
    plot(ah(2),nd*p3.x, p3.f, 'b+'); plot(ah(2),nd*(p3.x+A), p3.f+p3.s*A, 'b-'); drawnow
  end
  if wp(p3) || j >= LIMIT, break; end                                    % done?
  p0 = p1; p1 = p3;                                  % move points back one unit
  p3.x = p0.x + minCubic(p1.x-p0.x, p1.f-p0.f, p0.s, p1.s, 1);    % cubic extrap
end
while 1                                % keep interpolating as long as necessary
  if isnan(p3.f+p3.s) || isinf(p3.f+p3.s); p2 = p1; break; end % if final extrap failed
  if p1.f > p3.f, p2 = p3; else p2 = p1; end           % make p2 the best so far
  if wp(p2) > 1 || j >= LIMIT, break; end                                % done?
  p2.x = p1.x + minCubic(p3.x-p1.x, p3.f-p1.f, p1.s, p3.s, 0);    % cubic interp
  ok = 0;
  while ~ok && j < LIMIT;   % until function successfully evaluated or j = LIMIT
    try                                     % try to evaluate objective function
       j = j+1; [p2.f p2.df] = f(x0+p2.x*d); p2.s = p2.df'*d; ok = 1;
       if isnan(p2.f+p2.s) || isinf(p2.f+p2.s)
           error('Objective function returned Inf or NaN','');
       end
    catch                             % failed to successfully evaluate function
       if p.verbosity > 1, printf('\n'); warning(lasterr); end % warn or silence
       p2.x = (p1.x+p2.x)/2; ok = 0; if LIMIT == j; p2 = p1; end
    end
  end
  if p.verbosity > 2
    ah = ahandles(p); hold(ah(2),'on');
    plot(ah(2),nd*p2.x, p2.f, 'r+'); plot(ah(2),nd*(p2.x+A), p2.f+p2.s*A, 'r'); drawnow
  end
  if wp(p2) > -1 && p2.s > 0 || wp(p2) < -1, p3 = p2; else p1 = p2; end % bracket
end
x = x0+p2.x*d; fx = p2.f; df = p2.df; a = p2.x;        % return the value found
if p.length < 0, i = i+j; else i = i+1; end % count func evals or line searches
if p.verbosity, printf('%s %6i;  value %4.6e\r', p.S, i, fx); end
if wp(p2) < 2, i = -i; end                                   % indicate faliure
if p.verbosity > 2
  ah = ahandles(p); hold(ah(1),'on'); hold(ah(2),'on');
  if i>0, plot(ah(2),norm(d)*p2.x, fx, 'go'); end
  plot(ah(1),abs(i), fx, '+'); drawnow;
end

function z = minCubic(x, df, s0, s1, extr)   % minimizer of approximating cubic
INT = 0.1; EXT = 5.0;                    % interpolate and extrapolation limits
A = -6*df+3*(s0+s1)*x; B = 3*df-(2*s0+s1)*x; z = -s0*x*x/(B+sqrt(B*B-A*s0*x));
if extr                                                 % are we extrapolating?
  if ~isreal(z) | ~isfinite(z) | z < x | z > x*EXT, z = EXT*x; end  % fix bad z
  z = max(z, (1+INT)*x);                          % extrapolate by at least INT
else                                               % else, we are interpolating
  if ~isreal(z) | ~isfinite(z) | z < 0 | z > x, z = x/2; end;       % fix bad z
  z = min(max(z, INT*x), (1-INT)*x);    % at least INT away from the boundaries
end

function y = wp(p, SIG, RHO)
persistent a b c sig rho;
if nargin == 3    % if three arguments, then set up the Wolfe-Powell conditions
  a = RHO*p.s; b = p.f; c = -SIG*p.s; sig = SIG; rho = RHO; y = 0;
else
  if p.f > a*p.x+b                                  % function value too large?
    if a > 0, y = -1; else y = -2; end
  else
    if p.s < -c, y = 0; elseif p.s > c, y = 1; else y = 2; end
%   if sig*abs(p.s) > c, a = rho*p.s; b = p.f-a*p.x; c = sig*abs(p.s); end
  end
end

function [fx, dfx] = f(varargin)
persistent F p;
if nargout == 0
  p = varargin; if ischar(p{1}), F = str2func(p{1}); else F = p{1}; end
else
  [fx, dfx] = F(rewrap(p{2}, varargin{1}), p{3:end}); dfx = unwrap(dfx);
end

function v = unwrap(s)   % extract num elements of s (any type) into v (vector)
v = [];
if isnumeric(s)
  v = s(:);                        % numeric values are recast to column vector
elseif isstruct(s)
  v = unwrap(struct2cell(orderfields(s))); % alphabetize, conv to cell, recurse
elseif iscell(s)                                      % cell array elements are
  for i = 1:numel(s), v = [v; unwrap(s{i})]; end         % handled sequentially
end                                                   % other types are ignored

function [s v] = rewrap(s, v)    % map elements of v (vector) onto s (any type)
if isnumeric(s)
  if numel(v) < numel(s)
    error('The vector for conversion contains too few elements')
  end
  s = reshape(v(1:numel(s)), size(s));            % numeric values are reshaped
  v = v(numel(s)+1:end);                        % remaining arguments passed on
elseif isstruct(s)
  [s p] = orderfields(s); p(p) = 1:numel(p);      % alphabetize, store ordering
  [t v] = rewrap(struct2cell(s), v);                 % convert to cell, recurse
  s = orderfields(cell2struct(t,fieldnames(s),1),p);  % conv to struct, reorder
elseif iscell(s)
  for i = 1:numel(s)             % cell array elements are handled sequentially
    [s{i} v] = rewrap(s{i}, v);
  end
end                                             % other types are not processed

function printf(varargin)
fprintf(varargin{:}); if exist('fflush','builtin') fflush(stdout); end

function ah = ahandles(p)
persistent fh
if isfield(p,'fh') && ~isempty(p.fh) && ishandle(p.fh); fh = p.fh;
elseif isempty(fh) || ~ishandle(fh); fh = figure;
end
ah(1) = subplot(2,1,1,'Parent',fh); ah(2) = subplot(2,1,2,'Parent',fh);
\end{lstlisting}
