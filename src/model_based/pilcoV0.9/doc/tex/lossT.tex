
% This LaTeX was auto-generated from an M-file by MATLAB.
% To make changes, update the M-file and republish this document.



    
    

\subsection*{lossT.m} 

\begin{par}
\textbf{Summary:} Test derivatives of cost functions. It is assumed that the cost function computes (at least) the mean and the variance of the cost for a Gaussian distributed input $x\sim\mathcal N(m,s)$
\end{par} \vspace{1em}
\begin{verbatim}function [dd dy dh] = lossT(deriv, policy, m, s, delta)\end{verbatim}
\begin{par}
\textbf{Input arguments:}
\end{par} \vspace{1em}
\begin{verbatim}deriv    desired derivative. options:
      (i)   'dMdm' - derivative of the mean of the predicted cost
             wrt the mean of the input distribution
      (ii)  'dMds' - derivative of the mean of the predicted cost
             wrt the variance of the input distribution
      (iii) 'dSdm' - derivative of the variance of the predicted cost
             wrt the mean of the input distribution
      (iv)  'dSds' - derivative of the variance of the predicted cost
             wrt the variance of the input distribution
      (v)   'dCdm' - derivative of inv(s)*(covariance of the input and the
             predicted cost) wrt the mean of the input distribution
      (vi)  'dCds' - derivative of inv(s)*(covariance of the input and the
             predicted cost) wrt the variance of the input distribution
cost     cost structure
  .fcn   function handle to cost
  .\ensuremath{<}\ensuremath{>}    other fields that are passed on to the cost
m        mean of the input distribution
s        covariance of the input distribution
delta    (optional) finite difference parameter. Default: 1e-4\end{verbatim}
\begin{par}
\textbf{Output arguments:}
\end{par} \vspace{1em}
\begin{verbatim}dd         relative error of analytical vs. finite difference gradient
dy         analytical gradient
dh         finite difference gradient\end{verbatim}
\begin{par}
Copyright (C) 2008-2013 by Marc Deisenroth, Andrew McHutchon, Joe Hall, and Carl Edward Rasmussen.
\end{par} \vspace{1em}
\begin{par}
Last modified: 2013-05-30
\end{par} \vspace{1em}

\begin{lstlisting}
function [d dy dh] = lossT(deriv, cost, m, S, delta)
\end{lstlisting}


\subsection*{Code} 


\begin{lstlisting}
% create a default test if no input arguments are given
if nargin == 0;
  D = 4;
  m = randn(4,1);
  S = randn(4); S = S*S';
  cost.z = randn(4,1);
  W = randn(4); W = W*W';
  cost.W = W;
  cost.fcn = @lossQuad;
  deriv = 'dLdm';
end
D = length(m); if nargout < 5; delta = 1e-4; end

% check derivatives
switch deriv
  case {'dLdm', 'dMdm'}
    [d dy dh] = checkgrad(@losstest01, m, delta, cost, S);

  case {'dLds', 'dMds'}
    [d dy dh] = checkgrad(@losstest02, S(tril(ones(D))==1), delta, cost, m);

  case 'dSdm'
    [d dy dh] = checkgrad(@losstest03, m, delta, cost, S);

  case 'dSds'
    [d dy dh] = checkgrad(@losstest04, S(tril(ones(D))==1), delta, cost, m);

  case {'dCdm', 'dVdm'}
    [d dy dh] = checkgrad(@losstest05, m, delta, cost, S);

  case {'dCds', 'dVds'}
    [d dy dh] = checkgrad(@losstest06, S(tril(ones(D))==1), delta, cost, m);
end
\end{lstlisting}

\begin{lstlisting}
function [f, df] = losstest01(m, cost, S)                   % dLdm

[L dLdm] = cost.fcn(cost, m, S);

f = L; df = dLdm;


function [f, df] = losstest02(s, cost, m)                   % dLds

d = length(m);
ss(tril(ones(d))==1) = s; ss = reshape(ss,d,d); ss = ss + ss' - diag(diag(ss));

[L dLdm dLds] = cost.fcn(cost, m, ss);

f = L; df = dLds; df = 2*df-diag(diag(df)); df = df(tril(ones(d))==1);


function [f, df] = losstest03(m, cost, S)                   % dSdm

[L dLdm dLds S dSdm] = cost.fcn(cost, m, S);

f = S; df = dSdm;


function [f, df] = losstest04(s, cost, m)                   % dSds

d = length(m);
ss(tril(ones(d))==1) = s; ss = reshape(ss,d,d); ss = ss + ss' - diag(diag(ss));

[L dLdm dLds S dSdm dSds] = cost.fcn(cost, m, ss);

f = S; df = dSds; df = 2*df-diag(diag(df)); df = df(tril(ones(d))==1);


function [f, df] = losstest05(m, cost, S)                   % dCdm

[L dLdm dLds S dSdm dSds C dCdm] = cost.fcn(cost, m, S);

f = C; df = dCdm;


function [f, df] = losstest06(s, cost, m)                   % dCds

d = length(m);
ss(tril(ones(d))==1) = s; ss = reshape(ss,d,d); ss = ss + ss' - diag(diag(ss));

[L dLdm dLds S dSdm dSds C dCdm dCds] = cost.fcn(cost, m, ss);

f = C;
dCds = reshape(dCds,d,d,d); df = zeros(d,d*(d+1)/2);
for i=1:d;
  dCdsi = squeeze(dCds(i,:,:)); dCdsi = dCdsi+dCdsi'-diag(diag(dCdsi));
  df(i,:) = dCdsi(tril(ones(d))==1);
end;
\end{lstlisting}
